\documentclass[a4paper,12pt]{article}
\usepackage[utf8]{inputenc}
\usepackage{amsmath}
\usepackage{amssymb}
\usepackage{dsfont}

%opening
\title{Anisotropic Media Formulas from ``Landau Lifschitz -- Elektrodynamik der Kontinua''}

\newcommand{\vct}[1]{{\bf #1}}

\begin{document}

\maketitle

All units are in cgs. (TODO: convert to SI.)

\section{Energy dissipation (paragraph 96)}
\begin{align}
  Q &= \frac{i \omega}{16 \pi} 
  \left[
    (\varepsilon^\ast_{ik} - \varepsilon_{ki})E_i E_k^\ast
    + (\mu^\ast_{ik} - \mu_{ki})H_i H_k^\ast
  \right]\,.
\end{align}

Useful either at every surface or in grin media at
every integration point.

\section{Formal plane wave (paragraph 83)}
\begin{align}
 \exp(i \vct{k} \vct{r}) &= \exp(i \vct{k}^\prime \vct{r})
 \exp(-\vct{k}^{\prime\prime} \vct{r})\,,
\end{align}
where $\vct{k} = \vct{k}^\prime + i \vct{k}^{\prime\prime}$.

\section{Eikonal (paragraph 85)}

\begin{align}
 \phi &= a \exp(i \psi)\,,\\
 \frac{\partial \psi}{\partial t} &= -\omega\,,\quad
 \vct{\nabla} \psi = \vct{k}\,,\\
 \psi &= -\omega t + \frac{\omega}{c} \psi_1(x, y, z)\,.
\end{align}
From this it is:
\begin{align}
 \vct{\nabla}\psi_1 &= \vct{n}\text{ with }\vct{k} = \frac{\omega}{c}\vct{n}\,.
\end{align}
Therefore:
\begin{align}
 \psi_1 &= \int_A^B \vct{n}\text{d}\vct{l} \stackrel{\text{isotropic}}{=} \int_A^B n \text{d}l\,.
\end{align}
(For isotropic: $\vct{n}$ points in direction of $\text{d}\vct{l}$
and $\vct{n}^2 = n^2$ is the square of the refractive index.)

\section{Poynting Vector (paragraph 85)}
Represents the ray direction in optics.
\begin{align}
 \bar{\vct{S}} &= I \vct{l}\text{ and }\vct{\nabla}(I \vct{l}) = 0\,.\\
 \bar{\vct{S}} &= \mathfrak{R} 
 \left(
 \frac{c}{4\pi} \frac{1}{2} 
 \vct{E} \times \vct{H}^\ast
 \right)\,.
\end{align}
Also group $\tfrac{\partial \omega}{\partial \vct{k}}$ 
velocity represents ray direction.

\section{Dispersion for k (paragraph 97)}

\begin{align}
 (n^2 \delta_{ik} - n_i n_k - \varepsilon_{ik}) E_k &= 0\,,\\
 f(\vct{k}, \omega) = \det(n^2 \delta_{ik} - n_i n_k - \varepsilon_{ik}) &= 0\,.
\end{align}

\section{Introducing ray vector s (paragraph 97)}

Direction fixed by Poynting vector or group velocity.

\begin{align}
 \vct{s} &\sim \bar{\vct{S}} \sim \frac{\partial \omega}{\partial \vct{k}}\,.
\end{align}

Group velocity given by:
\begin{align}
 \frac{\partial \omega}{\partial \vct{k}} &= - 
 \frac{\frac{\partial f}{\partial \vct{k}}}
 {\frac{\partial f}{\partial \omega}}\,.
\end{align}


Introduction into Eikonal

\begin{align}
 \psi_1 &= \int_A^B \vct{n} \text{d}\vct{l}
 = \int_A^B \vct{n}\cdot\frac{\vct{s}}{s} \text{d}l\,.
\end{align}

Absolute value is fixed by demanding that
\begin{align}
 \vct{n}\cdot\vct{s} &= 1\,,
\end{align}
such that
\begin{align}
 \psi_1 &= \int_A^B \frac{\text{d}l}{s}\,.
\end{align}

\section{Dual Dispersion for s (paragraph 97)}

\begin{align}
 \det(s^2 \delta_{ik} - s_i s_k - (\varepsilon^{-1})_{ik}) &= 0\,.
\end{align}

Notice, this works also for complex valued $\varepsilon_{ij}$. $\vct{s}$ is then also complex valued
but points in the direction of $\bar{\vct{S}}$.

\section{Fermats principle and anisotropic media}

Let $\psi_1$ be real and demand that $\psi_1$ is minimal along the path the wave travels,
\begin{align}
 \psi_1 &= \int_A^B \vct{n}(\vct{x}) \text{d} \vct{x} = \text{min}\,.
\end{align}
Then the Lagrangian after introduction of a curve parameter $\tau$ is given by
\begin{align}
 L &= \vct{n}(\vct{x}(\tau)) \frac{\text{d}\vct{x}}{\text{d}\tau} = n_\ell \dot{x}_\ell\,.
\end{align}
Then
\begin{align}
 \frac{\partial L}{\partial x_k} &= (\partial_k n_\ell) \dot{x}_\ell\,,\\
 p_k = \frac{\partial L}{\partial \dot{x}_k} &= n_\ell \delta_{\ell k} = n_k\,. 
\end{align}
Therefore
\begin{align}
 \frac{\partial L}{\partial x_k} - \frac{\text{d}}{\text{d}\tau}\frac{\partial L}{\partial \dot{x}_k} &= 0\,,
\end{align}
leads to
\begin{align}
 (\partial_k n_\ell) \dot{x}_\ell - \frac{\text{d} n_k}{\text{d} \tau} &= 0\,.
\end{align}
This can further be simplified to
\begin{align}
 [(\partial_k n_\ell) - (\partial_\ell n_k)] \dot{x}_\ell &= 0\,.
\end{align}



\end{document}
